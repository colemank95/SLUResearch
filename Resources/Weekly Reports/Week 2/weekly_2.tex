%----------------------------------------------------------------------------------------
%	PACKAGES AND OTHER DOCUMENT CONFIGURATIONS
%----------------------------------------------------------------------------------------

\documentclass[paper=a4, fontsize=12pt]{scrartcl} % A4 paper and 12pt font size

\usepackage[T1]{fontenc} % Use 8-bit encoding that has 256 glyphs
\usepackage{fourier} % Use the Adobe Utopia font for the document - comment this line to return to the LaTeX default
\usepackage[english]{babel} % English language/hyphenation
\usepackage{amsmath,amsfonts,amsthm} % Math packages

\usepackage{sectsty} % Allows customizing section commands
\usepackage{hyperref} % Allows usage of hyperlinks
\allsectionsfont{\centering \normalfont\scshape} % Make all sections centered, the default font and small caps

\usepackage{fancyhdr} % Custom headers and footers
\pagestyle{fancyplain} % Makes all pages in the document conform to the custom headers and footers
\fancyhead{} % No page header - if you want one, create it in the same way as the footers below
\fancyfoot[L]{} % Empty left footer
\fancyfoot[C]{} % Empty center footer
\fancyfoot[R]{\thepage} % Page numbering for right footer
\renewcommand{\headrulewidth}{0pt} % Remove header underlines
\renewcommand{\footrulewidth}{0pt} % Remove footer underlines
\setlength{\headheight}{13.6pt} % Customize the height of the header

\numberwithin{equation}{section} % Number equations within sections (i.e. 1.1, 1.2, 2.1, 2.2 instead of 1, 2, 3, 4)
\numberwithin{figure}{section} % Number figures within sections (i.e. 1.1, 1.2, 2.1, 2.2 instead of 1, 2, 3, 4)
\numberwithin{table}{section} % Number tables within sections (i.e. 1.1, 1.2, 2.1, 2.2 instead of 1, 2, 3, 4)

\setlength\parindent{0pt} % Removes all indentation from paragraphs - comment this line for an assignment with lots of text

%----------------------------------------------------------------------------------------
%	TITLE SECTION
%----------------------------------------------------------------------------------------

\newcommand{\horrule}[1]{\rule{\linewidth}{#1}} % Create horizontal rule command with 1 argument of height

\title{	
\normalfont \normalsize 
\textsc{Saint Louis University, Department of Computer Science} \\ [25pt] % Your university, school and/or department name(s)
\horrule{0.5pt} \\[0.4cm] % Thin top horizontal rule
\huge Weekly Update 2\\ % The assignment title
\horrule{2pt} \\[0.5cm] % Thick bottom horizontal rule
}

\author{Kyle Coleman} % Your name

\date{September 17 - 23} % Today's date or a custom date

\begin{document}

\maketitle % Print the title

%----------------------------------------------------------------------------------------
%	Weekly Progress 
%----------------------------------------------------------------------------------------

\section{Weekly Progress}




%------------------------------------------------

\subsection{OpenWhisk Functions}
\begin{enumerate}
	\item Built a function for starting face-recognition process
    \item Built a function to make a DB query
	\item Built a function to make a HTTP post
\end{enumerate}


%------------------------------------------------

%----------------------------------------------------------------------------------------
%	Future Work 
%----------------------------------------------------------------------------------------

\section{Next Weeks Goals}

%------------------------------------------------

\subsection{OpenWhisk + Flask Interoperability}
\begin{itemize}
	\item Actually sending photos from Flask to server. Do I need some kind of program running on server to receive?
	\item Get the returned data to display on user device
	\item Possibly make Flask page for displaying info?
	
\end{itemize}

%------------------------------------------------

\subsection{Database}
\begin{enumerate}
\item Work on building DB
\end{enumerate}

%----------------------------------------------------------------------------------------

%----------------------------------------------------------------------------------------
%	Roadblocks
%---------------------------------------------------------------------------------------

\section{Roadblocks}
\begin{enumerate}
\item I'm not sure how to get the photos from aggregator to OpenWhisk server 
\end{enumerate}
%------------------------------------------------

None right now

\end{document}