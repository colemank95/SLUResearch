%----------------------------------------------------------------------------------------
%	PACKAGES AND OTHER DOCUMENT CONFIGURATIONS
%----------------------------------------------------------------------------------------

\documentclass[paper=a4, fontsize=12pt]{scrartcl} % A4 paper and 12pt font size

\usepackage[T1]{fontenc} % Use 8-bit encoding that has 256 glyphs
\usepackage{fourier} % Use the Adobe Utopia font for the document - comment this line to return to the LaTeX default
\usepackage[english]{babel} % English language/hyphenation
\usepackage{amsmath,amsfonts,amsthm} % Math packages

\usepackage{sectsty} % Allows customizing section commands
\usepackage{hyperref} % Allows usage of hyperlinks
\allsectionsfont{\centering \normalfont\scshape} % Make all sections centered, the default font and small caps

\usepackage{fancyhdr} % Custom headers and footers
\pagestyle{fancyplain} % Makes all pages in the document conform to the custom headers and footers
\fancyhead{} % No page header - if you want one, create it in the same way as the footers below
\fancyfoot[L]{} % Empty left footer
\fancyfoot[C]{} % Empty center footer
\fancyfoot[R]{\thepage} % Page numbering for right footer
\renewcommand{\headrulewidth}{0pt} % Remove header underlines
\renewcommand{\footrulewidth}{0pt} % Remove footer underlines
\setlength{\headheight}{13.6pt} % Customize the height of the header

\numberwithin{equation}{section} % Number equations within sections (i.e. 1.1, 1.2, 2.1, 2.2 instead of 1, 2, 3, 4)
\numberwithin{figure}{section} % Number figures within sections (i.e. 1.1, 1.2, 2.1, 2.2 instead of 1, 2, 3, 4)
\numberwithin{table}{section} % Number tables within sections (i.e. 1.1, 1.2, 2.1, 2.2 instead of 1, 2, 3, 4)

\setlength\parindent{0pt} % Removes all indentation from paragraphs - comment this line for an assignment with lots of text

%----------------------------------------------------------------------------------------
%	TITLE SECTION
%----------------------------------------------------------------------------------------

\newcommand{\horrule}[1]{\rule{\linewidth}{#1}} % Create horizontal rule command with 1 argument of height

\title{	
\normalfont \normalsize 
\textsc{Saint Louis University, Department of Computer Science} \\ [25pt] % Your university, school and/or department name(s)
\horrule{0.5pt} \\[0.4cm] % Thin top horizontal rule
\huge Weekly Update 1\\ % The assignment title
\horrule{2pt} \\[0.5cm] % Thick bottom horizontal rule
}

\author{Kyle Coleman} % Your name

\date{September 10 - 16} % Today's date or a custom date

\begin{document}

\maketitle % Print the title

%----------------------------------------------------------------------------------------
%	Weekly Progress 
%----------------------------------------------------------------------------------------

\section{Weekly Progress}




%------------------------------------------------

\subsection{One Page White Paper}
\begin{enumerate}
	\item Drafted the paper
    \item Cut down to one page + editing
    \item Reformatted the paper to match ACM standards
    \item Finalized the figure by including something 	 signifying how parents upload photos to the DB
    \item Submit to WOSC 2017 at \href{https://middleware17wosc.hotcrp.com}{the WOSC Middleware Submission Site}
\end{enumerate}

\subsection{OpenWhisk}
\begin{enumerate}
	\item Installed a local clone of OpenWhisk
    \item Set up function environment
    \item Worked on interoperability with Flask
\end{enumerate}

\subsection{Database}
\begin{enumerate}
	\item Drew schema for the diagram
    \item Looked into mySQL for use
\end{enumerate}

%------------------------------------------------

%----------------------------------------------------------------------------------------
%	Future Work 
%----------------------------------------------------------------------------------------

\section{Next Weeks Goals}

%------------------------------------------------

\subsection{OpenWhisk + Flask Interoperability}
\begin{itemize}
	\item Finish any work relating to OpenWhisk functions and Flask 
		\begin{itemize}
		\item Function for detecting pictures being sent to OpenWhisk 
		\item Function for initiating a DB Query
        \item Function for sending data back to users
		\end{itemize} 
\end{itemize}

%------------------------------------------------

\subsection{Database}
\begin{enumerate}
\item If I haven't finished the DB setup, then finish it 
\end{enumerate}

%----------------------------------------------------------------------------------------

%----------------------------------------------------------------------------------------
%	Roadblocks
%---------------------------------------------------------------------------------------

\section{Roadblocks}

%------------------------------------------------

None right now

\end{document}