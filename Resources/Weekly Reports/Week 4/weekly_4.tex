%----------------------------------------------------------------------------------------
%	PACKAGES AND OTHER DOCUMENT CONFIGURATIONS
%----------------------------------------------------------------------------------------

\documentclass[paper=a4, fontsize=12pt]{scrartcl} % A4 paper and 12pt font size

\usepackage[T1]{fontenc} % Use 8-bit encoding that has 256 glyphs
\usepackage{fourier} % Use the Adobe Utopia font for the document - comment this line to return to the LaTeX default
\usepackage[english]{babel} % English language/hyphenation
\usepackage{amsmath,amsfonts,amsthm} % Math packages

\usepackage{sectsty} % Allows customizing section commands
\usepackage{hyperref} % Allows usage of hyperlinks
\allsectionsfont{\centering \normalfont\scshape} % Make all sections centered, the default font and small caps

\usepackage{fancyhdr} % Custom headers and footers
\pagestyle{fancyplain} % Makes all pages in the document conform to the custom headers and footers
\fancyhead{} % No page header - if you want one, create it in the same way as the footers below
\fancyfoot[L]{} % Empty left footer
\fancyfoot[C]{} % Empty center footer
\fancyfoot[R]{\thepage} % Page numbering for right footer
\renewcommand{\headrulewidth}{0pt} % Remove header underlines
\renewcommand{\footrulewidth}{0pt} % Remove footer underlines
\setlength{\headheight}{13.6pt} % Customize the height of the header

\numberwithin{equation}{section} % Number equations within sections (i.e. 1.1, 1.2, 2.1, 2.2 instead of 1, 2, 3, 4)
\numberwithin{figure}{section} % Number figures within sections (i.e. 1.1, 1.2, 2.1, 2.2 instead of 1, 2, 3, 4)
\numberwithin{table}{section} % Number tables within sections (i.e. 1.1, 1.2, 2.1, 2.2 instead of 1, 2, 3, 4)

\setlength\parindent{0pt} % Removes all indentation from paragraphs - comment this line for an assignment with lots of text

%----------------------------------------------------------------------------------------
%	TITLE SECTION
%----------------------------------------------------------------------------------------

\newcommand{\horrule}[1]{\rule{\linewidth}{#1}} % Create horizontal rule command with 1 argument of height

\title{	
\normalfont \normalsize 
\textsc{Saint Louis University, Department of Computer Science} \\ [25pt] % Your university, school and/or department name(s)
\horrule{0.5pt} \\[0.4cm] % Thin top horizontal rule
\huge Weekly Update 4\\ % The assignment title
\horrule{2pt} \\[0.5cm] % Thick bottom horizontal rule
}

\author{Kyle Coleman} % Your name

\date{October 2 - October 8} % Today's date or a custom date

\begin{document}

\maketitle % Print the title

%----------------------------------------------------------------------------------------
%	Weekly Progress 
%----------------------------------------------------------------------------------------

\section{Weekly Progress}




%------------------------------------------------

\subsection{Rebuild 2.0}
\begin{enumerate}
	\item It turns out that the Docker images built will only work properly on Ubuntu 14.04 (this isn't mentioned in OpenWhisk documentation)
    \item Rebuilt the VM on 14.04 and rebuilt OpenWhisk (again)
\end{enumerate}


\subsection{OpenWhisk}
\begin{enumerate}
	\item Now that OpenWhisk is working properly I can create actions. (Will show a semi-functioning one at meeting)
	\item The face-recognition action is ready to go. I only need to find out how to allow actions to reference files on the same machine (will try to explain this better in person)
	\item I know what kind of DB Queries to incorporate into action
\end{enumerate}


%------------------------------------------------

%----------------------------------------------------------------------------------------
%	Future Work 
%----------------------------------------------------------------------------------------

\section{Next Weeks Goals}

%------------------------------------------------

\subsection{OpenWhisk + Flask Interoperability}
\begin{itemize}
	\item Biggest focus this week will be sending photos from Flask to OW
	\item Since it is running locally, i should be able to just send photos to my machine	
\end{itemize}

%------------------------------------------------

\subsection{Testing}
\begin{itemize}
	\item If the photo sending stuff is easy to do, i can begin this this week. 
\end{itemize}

%----------------------------------------------------------------------------------------

%----------------------------------------------------------------------------------------
%	Roadblocks
%---------------------------------------------------------------------------------------

\section{Roadblocks}
\begin{enumerate}
	\item  The re-rebuild held me up a bit. OpenWhisk takes a very long time to build (especially on a slow internet connection)
	\item  I'm not entirely sure how to get actions to utilize photos on the local machine but I will figure it out
\end{enumerate}
%------------------------------------------------

\end{document}